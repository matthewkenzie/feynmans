\documentclass{standalone}
\usepackage{axodraw2}
\usepackage{ifthen}
\newboolean{uprightparticles}
\setboolean{uprightparticles}{false}
\newboolean{pdflatex}
\setboolean{pdflatex}{true}
\input{lhcb-symbols-def}
\begin{document}
\begin{axopicture}(190,140)

  % A meson                         
  \Text(30,92)[lb]{$\bquark$}       % quark Label
  \Line[arrow](25,90)(85,90)        % quark Line
  \Text(30,48)[lt]{$\uquarkbar$}    % anti-quark Label
  \Line[arrow](85,50)(25,50)        % anti-quark Line
  \Text(18,70)[r]{$\Bm$}            % Label
  \GOval(25,70)(5,20)(90){0.7}      % Bound State
                                    
  % Internal W line                 
  \Text(113,86){\small{$\Wm$}}      % Label
  \Photon(85,90)(125,70){-2}{5}     % Line
  \Vertex(85,90){2}                 % Start Vertex
  \Vertex(125,70){2}                % End Vertex
                                    
  % B meson                         
  \Text(160,129)[rb]{$\uquark$}     % quark Label
  \Line[arrow](85,90)(165,130)      % quark Line
  \Text(160,84)[rt]{$\cquarkbar$}   % anti-quark Label
  \Line[arrow](165,90)(125,70)      % anti-quark Line
  \Text(172,110)[l]{$\Dzb$}         % Label
  \GOval(165,110)(5,20)(90){0.7}    % Bound State
                                    
  % C meson                         
  \Text(160,56)[rb]{$\squark$}      % quark Label
  \Line[arrow](125,70)(165,50)      % quark Line
  \Text(160,11)[rt]{$\uquarkbar$}   % anti-quark Label
  \Line[arrow](165,10)(85,50)       % anti-quark Line
  \Text(172,30)[l]{$\Km$}           % Label
  \GOval(165,30)(5,20)(90){0.7}     % Bound State
                                    
\end{axopicture}
\end{document}
