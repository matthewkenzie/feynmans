\documentclass{standalone}
\usepackage{axodraw2}
\usepackage{ifthen}
\newboolean{uprightparticles}
\setboolean{uprightparticles}{false}
\newboolean{pdflatex}
\setboolean{pdflatex}{true}
\input{lhcb-symbols-def}
\begin{document}
\begin{axopicture}(190,140)

  % A meson                                      
  \Text(30,92)[lb]{$\bquarkbar$}                 % anti-quark Label
  \Line[arrow](85,90)(25,90)                     % anti-quark Line
  \Text(30,48)[lt]{$\squark$}                    % quark Label
  \Line[arrow](25,50)(85,50)                     % quark Line
  \Text(18,70)[r]{$\Bs$}                         % Label
  \GOval(25,70)(5,20)(90){0.7}                   % Bound State
                                                 
  % Loop                                         
  \Text(105,75){\small{$\Wp$}}                   % W Label
  \PhotonArc(108,86)(23,171,312){-3}{8}          % W Line
                                                 % fermion Label
  \Arc[arrow,arrowpos=0.65](102,74)(23,350,45)   % fermion Line1
  \Arc[arrow,arrowpos=0.55](102,74)(23,45,135)   % fermion Line2
  \Vertex(85,90){2}                              % Start Vertex
  \Vertex(125,70){2}                             % End Vertex
  % Gluon                                        
  \Gluon(113,95)(135,110){-3.5}{3}               % gluon Line
  \Vertex(113,95){2}                             % Start Vertex
  \Vertex(135,110){2}                            % End Vertex
  % B meson                                      
  \Text(160,129)[rb]{$\cquarkbar$}               % anti-quark Label
  \Line[arrow](165,130)(135,110)                 % anti-quark Line
  \Text(160,91)[rt]{$\cquark$}                   % quark Label
  \Line[arrow](135,110)(165,90)                  % quark Line
  \Text(172,110)[l]{$\jpsi$}                     % Label
  \GOval(165,110)(5,20)(90){0.7}                 % Bound State
                                                 
  % C meson                                      
  \Text(160,56)[rb]{$\squarkbar$}                % anti-quark Label
  \Line[arrow](165,50)(125,70)                   % anti-quark Line
  \Text(160,11)[rt]{$\squark$}                   % quark Label
  \Line[arrow](85,50)(165,10)                    % quark Line
  \Text(172,30)[l]{$\phi$}                       % Label
  \GOval(165,30)(5,20)(90){0.7}                  % Bound State
                                                 
\end{axopicture}
\end{document}
