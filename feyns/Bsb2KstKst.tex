\documentclass{standalone}
\usepackage{axodraw2}
\usepackage{ifthen}
\newboolean{uprightparticles}
\setboolean{uprightparticles}{false}
\newboolean{pdflatex}
\setboolean{pdflatex}{true}
\input{lhcb-symbols-def}
\begin{document}
\begin{axopicture}(190,140)

  % A meson                                       
  \Text(30,92)[lb]{$\bquark$}                     % quark Label
  \Line[arrow](25,90)(64,90)                      % quark Line
  \Text(30,48)[lt]{$\squarkbar$}                  % anti-quark Label
  \Line[arrow](85,50)(25,50)                      % anti-quark Line
  \Text(18,70)[r]{$\Bsb$}                         % Label
  \GOval(25,70)(5,20)(90){0.7}                    % Bound State
                                                  
  % Loop                                          
  \Text(80,115){\small{$\Wm$}}                    % W Label
  \PhotonArc(87,87)(23,35,173){-2}{7}             % W Line
                                                  % fermion Label
  \Arc[arrow,arrowpos=0.60](83,103)(23,290,355)   % fermion Line1
  \Arc[arrow,arrowpos=0.60](83,103)(23,215,290)   % fermion Line2
  \Vertex(64,90){2}                               % Start Vertex
  \Vertex(106,100.5){2}                           % End Vertex
  % Gluon                                         
  \Gluon(97,84)(125,70){-3.5}{3}                  % gluon Line
  \Vertex(125,70){2}                              % Start Vertex
  \Vertex(97,84){2}                               % End Vertex
  % B meson                                       
  \Text(160,129)[rb]{$\squark$}                   % quark Label
  \Line[arrow](106,100.5)(165,130)                % quark Line
  \Text(160,84)[rt]{$\dquarkbar$}                 % anti-quark Label
  \Line[arrow](165,90)(125,70)                    % anti-quark Line
  \Text(172,110)[l]{$\Kstarzb$}                   % Label
  \GOval(165,110)(5,20)(90){0.7}                  % Bound State
                                                  
  % C meson                                       
  \Text(160,56)[rb]{$\dquark$}                    % quark Label
  \Line[arrow](125,70)(165,50)                    % quark Line
  \Text(160,11)[rt]{$\squarkbar$}                 % anti-quark Label
  \Line[arrow](165,10)(85,50)                     % anti-quark Line
  \Text(172,30)[l]{$\Kstarz$}                     % Label
  \GOval(165,30)(5,20)(90){0.7}                   % Bound State
                                                  
\end{axopicture}
\end{document}
