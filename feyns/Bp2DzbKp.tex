\documentclass{standalone}
\usepackage{axodraw2}
\usepackage{ifthen}
\newboolean{uprightparticles}
\setboolean{uprightparticles}{false}
\newboolean{pdflatex}
\setboolean{pdflatex}{true}
\input{lhcb-symbols-def}
\begin{document}
\begin{axopicture}(190,140)

  % A meson                          
  \Text(30,92)[lb]{$\bquarkbar$}     % anti-quark Label
  \Line[arrow](85,90)(25,90)         % anti-quark Line
  \Text(30,48)[lt]{$\uquark$}        % quark Label
  \Line[arrow](25,50)(85,50)         % quark Line
  \Text(18,70)[r]{$\Bp$}             % Label
  \GOval(25,70)(5,20)(90){0.7}       % Bound State
                                     
  % External W line                  
  \Text(105,108){\small{$\Wp$}}      % Label
  \Photon(85,90)(125,110){-2}{5}     % Line
  \Vertex(85,90){2}                  % Start Vertex
  \Vertex(125,110){2}                % End Vertex
                                     
  % B meson                          
  \Text(160,129)[rb]{$\squarkbar$}   % anti-quark Label
  \Line[arrow](165,130)(125,110)     % anti-quark Line
  \Text(160,91)[rt]{$\uquark$}       % quark Label
  \Line[arrow](125,110)(165,90)      % quark Line
  \Text(172,110)[l]{$\Kp$}           % Label
  \GOval(165,110)(5,20)(90){0.7}     % Bound State
                                     
  % C meson                          
  \Text(160,56)[rb]{$\cquarkbar$}    % anti-quark Label
  \Line[arrow](165,50)(85,90)        % anti-quark Line
  \Text(160,11)[rt]{$\uquark$}       % quark Label
  \Line[arrow](85,50)(165,10)        % quark Line
  \Text(172,30)[l]{$\Dzb$}           % Label
  \GOval(165,30)(5,20)(90){0.7}      % Bound State
                                     
\end{axopicture}
\end{document}
