\documentclass{standalone}
\usepackage{axodraw2}
\usepackage{ifthen}
\newboolean{uprightparticles}
\setboolean{uprightparticles}{false}
\newboolean{pdflatex}
\setboolean{pdflatex}{true}
%%%%%%%%%%%%%%%%%%%%%%%%%%%%%%%%%%%%%%%%%%%%%%%%%%%%%%%%%%%%%%%%%%%%%%%%
%%%                                                                    %
%%% !!!!!!!!!!!!!!!!!!! DO NOT EDIT THIS FILE !!!!!!!!!!!!!!!!!!!!!!!! %
%%%                                                                    %
%%% THE EB MAY OVERWRITE IT TO REFLECT LATEST CHANGES IN THE TEMPLATE  %
%%%                                                                    %
%%% You may define your own macros and packages in main.tex or add     %
%%% additional local files                                             %
%%%%%%%%%%%%%%%%%%%%%%%%%%%%%%%%%%%%%%%%%%%%%%%%%%%%%%%%%%%%%%%%%%%%%%%%
%%% ======================================================================
%%% Purpose: Standard LHCb aliases
%%% Author: Originally Ulrik Egede, adapted by Tomasz Skwarnicki for templates,
%%% rewritten by Chris Parkes
%%% Maintainer : Ulrik Egede (2010 - 2012)
%%% Maintainer : Rolf Oldeman (2012 - 2014)
%%% Maintainer : Patrick Koppenburg (2018--2020)
%%% =======================================================================
%%% To use this file outside the normal LHCb document environment, the
%%% following should be added in a preamble (before \begin{document}
%%%
%%%\usepackage{ifthen}
%%%\newboolean{uprightparticles}
%%%\setboolean{uprightparticles}{false} %Set true for upright particle symbols
\usepackage{xspace}
\usepackage{upgreek}


%%%%%%%%%%%%%%%%%%%%%%%%%%%%%%%%%%%%%%%%%%%%%%%%%%%%%%%%%%%%
%%%
%%% The following is to ensure that the template automatically can process
%%% this file.
%%%
%%% Add comments with at least three %%% preceding.
%%% Add new sections with one % preceding
%%% Add new subsections with two %% preceding
%%%
%%% For upper greek letters, Xires and Xiresbar will be the particles without the charge
%%% States with charge are called Xiz and Xim
%%%
%%%%%%%%%%%%%%%%%%%%%%%%%%%%%%%%%%%%%%%%%%%%%%%%%%%%%%%%%%%%

%%%%%%%%%%%%%
% Experiments
%%%%%%%%%%%%%
\def\lhcb   {\mbox{LHCb}\xspace}
\def\atlas  {\mbox{ATLAS}\xspace}
\def\cms    {\mbox{CMS}\xspace}
\def\alice  {\mbox{ALICE}\xspace}
\def\babar  {\mbox{BaBar}\xspace}
\def\belle  {\mbox{Belle}\xspace}
\def\belletwo {\mbox{Belle~II}\xspace}
\def\besiii {\mbox{BESIII}\xspace}
\def\cleo   {\mbox{CLEO}\xspace}
\def\cdf    {\mbox{CDF}\xspace}
\def\dzero  {\mbox{D0}\xspace}
\def\aleph  {\mbox{ALEPH}\xspace}
\def\delphi {\mbox{DELPHI}\xspace}
\def\opal   {\mbox{OPAL}\xspace}
\def\lthree {\mbox{L3}\xspace}
\def\sld    {\mbox{SLD}\xspace}
%%%\def\argus  {\mbox{ARGUS}\xspace}
%%%\def\uaone  {\mbox{UA1}\xspace}
%%%\def\uatwo  {\mbox{UA2}\xspace}
%%%\def\ux85 {\mbox{UX85}\xspace}
\def\cern {\mbox{CERN}\xspace}
\def\lhc    {\mbox{LHC}\xspace}
\def\lep    {\mbox{LEP}\xspace}
\def\tevatron {Tevatron\xspace}
\def\bfactories {\mbox{\B Factories}\xspace}
\def\bfactory   {\mbox{\B Factory}\xspace}
\def\upgradeone {\mbox{Upgrade~I}\xspace}
\def\upgradetwo {\mbox{Upgrade~II}\xspace}

%% LHCb sub-detectors and sub-systems

%%%\def\pu     {PU\xspace}
\def\velo   {VELO\xspace}
\def\rich   {RICH\xspace}
\def\richone {RICH1\xspace}
\def\richtwo {RICH2\xspace}
\def\ttracker {TT\xspace}
\def\intr   {IT\xspace}
\def\st     {ST\xspace}
\def\ot     {OT\xspace}
\def\herschel {\mbox{\textsc{HeRSCheL}}\xspace}
%%%\def\Tone   {T1\xspace}
%%%\def\Ttwo   {T2\xspace}
%%%\def\Tthree {T3\xspace}
%%%\def\Mone   {M1\xspace}
%%%\def\Mtwo   {M2\xspace}
%%%\def\Mthree {M3\xspace}
%%%\def\Mfour  {M4\xspace}
%%%\def\Mfive  {M5\xspace}
\def\spd    {SPD\xspace}
\def\presh  {PS\xspace}
\def\ecal   {ECAL\xspace}
\def\hcal   {HCAL\xspace}
%%%\def\bcm    {BCM\xspace}
\def\MagUp {\mbox{\em Mag\kern -0.05em Up}\xspace}
\def\MagDown {\mbox{\em MagDown}\xspace}

\def\ode    {ODE\xspace}
\def\daq    {DAQ\xspace}
\def\tfc    {TFC\xspace}
\def\ecs    {ECS\xspace}
\def\lone   {L0\xspace}
\def\hlt    {HLT\xspace}
\def\hltone {HLT1\xspace}
\def\hlttwo {HLT2\xspace}

%%% Upright (not slanted) Particles

\ifthenelse{\boolean{uprightparticles}}%
{\def\Palpha      {\ensuremath{\upalpha}\xspace}
 \def\Pbeta       {\ensuremath{\upbeta}\xspace}
 \def\Pgamma      {\ensuremath{\upgamma}\xspace}
 \def\Pdelta      {\ensuremath{\updelta}\xspace}
 \def\Pepsilon    {\ensuremath{\upepsilon}\xspace}
 \def\Pvarepsilon {\ensuremath{\upvarepsilon}\xspace}
 \def\Pzeta       {\ensuremath{\upzeta}\xspace}
 \def\Peta        {\ensuremath{\upeta}\xspace}
 \def\Ptheta      {\ensuremath{\uptheta}\xspace}
 \def\Pvartheta   {\ensuremath{\upvartheta}\xspace}
 \def\Piota       {\ensuremath{\upiota}\xspace}
 \def\Pkappa      {\ensuremath{\upkappa}\xspace}
 \def\Plambda     {\ensuremath{\uplambda}\xspace}
 \def\Pmu         {\ensuremath{\upmu}\xspace}
 \def\Pnu         {\ensuremath{\upnu}\xspace}
 \def\Pxi         {\ensuremath{\upxi}\xspace}
 \def\Ppi         {\ensuremath{\uppi}\xspace}
 \def\Pvarpi      {\ensuremath{\upvarpi}\xspace}
 \def\Prho        {\ensuremath{\uprho}\xspace}
 \def\Pvarrho     {\ensuremath{\upvarrho}\xspace}
 \def\Ptau        {\ensuremath{\uptau}\xspace}
 \def\Pupsilon    {\ensuremath{\upupsilon}\xspace}
 \def\Pphi        {\ensuremath{\upphi}\xspace}
 \def\Pvarphi     {\ensuremath{\upvarphi}\xspace}
 \def\Pchi        {\ensuremath{\upchi}\xspace}
 \def\Ppsi        {\ensuremath{\uppsi}\xspace}
 \def\Pomega      {\ensuremath{\upomega}\xspace}

 \def\PDelta      {\ensuremath{\Delta}\xspace}
 \def\PXi         {\ensuremath{\Xi}\xspace}
 \def\PLambda     {\ensuremath{\Lambda}\xspace}
 \def\PSigma      {\ensuremath{\Sigma}\xspace}
 \def\POmega      {\ensuremath{\Omega}\xspace}
 \def\PUpsilon    {\ensuremath{\Upsilon}\xspace}

 \def\PA      {\ensuremath{\mathrm{A}}\xspace}
 \def\PB      {\ensuremath{\mathrm{B}}\xspace}
 \def\PC      {\ensuremath{\mathrm{C}}\xspace}
 \def\PD      {\ensuremath{\mathrm{D}}\xspace}
 \def\PE      {\ensuremath{\mathrm{E}}\xspace}
 \def\PF      {\ensuremath{\mathrm{F}}\xspace}
 \def\PG      {\ensuremath{\mathrm{G}}\xspace}
 \def\PH      {\ensuremath{\mathrm{H}}\xspace}
 \def\PI      {\ensuremath{\mathrm{I}}\xspace}
 \def\PJ      {\ensuremath{\mathrm{J}}\xspace}
 \def\PK      {\ensuremath{\mathrm{K}}\xspace}
 \def\PL      {\ensuremath{\mathrm{L}}\xspace}
 \def\PM      {\ensuremath{\mathrm{M}}\xspace}
 \def\PN      {\ensuremath{\mathrm{N}}\xspace}
 \def\PO      {\ensuremath{\mathrm{O}}\xspace}
 \def\PP      {\ensuremath{\mathrm{P}}\xspace}
 \def\PQ      {\ensuremath{\mathrm{Q}}\xspace}
 \def\PR      {\ensuremath{\mathrm{R}}\xspace}
 \def\PS      {\ensuremath{\mathrm{S}}\xspace}
 \def\PT      {\ensuremath{\mathrm{T}}\xspace}
 \def\PU      {\ensuremath{\mathrm{U}}\xspace}
 \def\PV      {\ensuremath{\mathrm{V}}\xspace}
 \def\PW      {\ensuremath{\mathrm{W}}\xspace}
 \def\PX      {\ensuremath{\mathrm{X}}\xspace}
 \def\PY      {\ensuremath{\mathrm{Y}}\xspace}
 \def\PZ      {\ensuremath{\mathrm{Z}}\xspace}
 \def\Pa      {\ensuremath{\mathrm{a}}\xspace}
 \def\Pb      {\ensuremath{\mathrm{b}}\xspace}
 \def\Pc      {\ensuremath{\mathrm{c}}\xspace}
 \def\Pd      {\ensuremath{\mathrm{d}}\xspace}
 \def\Pe      {\ensuremath{\mathrm{e}}\xspace}
 \def\Pf      {\ensuremath{\mathrm{f}}\xspace}
 \def\Pg      {\ensuremath{\mathrm{g}}\xspace}
 \def\Ph      {\ensuremath{\mathrm{h}}\xspace}
 \def\Pi      {\ensuremath{\mathrm{i}}\xspace}
 \def\Pj      {\ensuremath{\mathrm{j}}\xspace}
 \def\Pk      {\ensuremath{\mathrm{k}}\xspace}
 \def\Pl      {\ensuremath{\mathrm{l}}\xspace}
 \def\Pm      {\ensuremath{\mathrm{m}}\xspace}
 \def\Pn      {\ensuremath{\mathrm{n}}\xspace}
 \def\Po      {\ensuremath{\mathrm{o}}\xspace}
 \def\Pp      {\ensuremath{\mathrm{p}}\xspace}
 \def\Pq      {\ensuremath{\mathrm{q}}\xspace}
 \def\Pr      {\ensuremath{\mathrm{r}}\xspace}
 \def\Ps      {\ensuremath{\mathrm{s}}\xspace}
 \def\Pt      {\ensuremath{\mathrm{t}}\xspace}
 \def\Pu      {\ensuremath{\mathrm{u}}\xspace}
 \def\Pv      {\ensuremath{\mathrm{v}}\xspace}
 \def\Pw      {\ensuremath{\mathrm{w}}\xspace}
 \def\Px      {\ensuremath{\mathrm{x}}\xspace}
 \def\Py      {\ensuremath{\mathrm{y}}\xspace}
 \def\Pz      {\ensuremath{\mathrm{z}}\xspace}
 \def\thebaroffset{0.0em}
}
{\def\Palpha      {\ensuremath{\alpha}\xspace}
 \def\Pbeta       {\ensuremath{\beta}\xspace}
 \def\Pgamma      {\ensuremath{\gamma}\xspace}
 \def\Pdelta      {\ensuremath{\delta}\xspace}
 \def\Pepsilon    {\ensuremath{\epsilon}\xspace}
 \def\Pvarepsilon {\ensuremath{\varepsilon}\xspace}
 \def\Pzeta       {\ensuremath{\zeta}\xspace}
 \def\Peta        {\ensuremath{\eta}\xspace}
 \def\Ptheta      {\ensuremath{\theta}\xspace}
 \def\Pvartheta   {\ensuremath{\vartheta}\xspace}
 \def\Piota       {\ensuremath{\iota}\xspace}
 \def\Pkappa      {\ensuremath{\kappa}\xspace}
 \def\Plambda     {\ensuremath{\lambda}\xspace}
 \def\Pmu         {\ensuremath{\mu}\xspace}
 \def\Pnu         {\ensuremath{\nu}\xspace}
 \def\Pxi         {\ensuremath{\xi}\xspace}
 \def\Ppi         {\ensuremath{\pi}\xspace}
 \def\Pvarpi      {\ensuremath{\varpi}\xspace}
 \def\Prho        {\ensuremath{\rho}\xspace}
 \def\Pvarrho     {\ensuremath{\varrho}\xspace}
 \def\Ptau        {\ensuremath{\tau}\xspace}
 \def\Pupsilon    {\ensuremath{\upsilon}\xspace}
 \def\Pphi        {\ensuremath{\phi}\xspace}
 \def\Pvarphi     {\ensuremath{\varphi}\xspace}
 \def\Pchi        {\ensuremath{\chi}\xspace}
 \def\Ppsi        {\ensuremath{\psi}\xspace}
 \def\Pomega      {\ensuremath{\omega}\xspace}
 \mathchardef\PDelta="7101
 \mathchardef\PXi="7104
 \mathchardef\PLambda="7103
 \mathchardef\PSigma="7106
 \mathchardef\POmega="710A
 \mathchardef\PUpsilon="7107
 \def\PA      {\ensuremath{A}\xspace}
 \def\PB      {\ensuremath{B}\xspace}
 \def\PC      {\ensuremath{C}\xspace}
 \def\PD      {\ensuremath{D}\xspace}
 \def\PE      {\ensuremath{E}\xspace}
 \def\PF      {\ensuremath{F}\xspace}
 \def\PG      {\ensuremath{G}\xspace}
 \def\PH      {\ensuremath{H}\xspace}
 \def\PI      {\ensuremath{I}\xspace}
 \def\PJ      {\ensuremath{J}\xspace}
 \def\PK      {\ensuremath{K}\xspace}
 \def\PL      {\ensuremath{L}\xspace}
 \def\PM      {\ensuremath{M}\xspace}
 \def\PN      {\ensuremath{N}\xspace}
 \def\PO      {\ensuremath{O}\xspace}
 \def\PP      {\ensuremath{P}\xspace}
 \def\PQ      {\ensuremath{Q}\xspace}
 \def\PR      {\ensuremath{R}\xspace}
 \def\PS      {\ensuremath{S}\xspace}
 \def\PT      {\ensuremath{T}\xspace}
 \def\PU      {\ensuremath{U}\xspace}
 \def\PV      {\ensuremath{V}\xspace}
 \def\PW      {\ensuremath{W}\xspace}
 \def\PX      {\ensuremath{X}\xspace}
 \def\PY      {\ensuremath{Y}\xspace}
 \def\PZ      {\ensuremath{Z}\xspace}
 \def\Pa      {\ensuremath{a}\xspace}
 \def\Pb      {\ensuremath{b}\xspace}
 \def\Pc      {\ensuremath{c}\xspace}
 \def\Pd      {\ensuremath{d}\xspace}
 \def\Pe      {\ensuremath{e}\xspace}
 \def\Pf      {\ensuremath{f}\xspace}
 \def\Pg      {\ensuremath{g}\xspace}
 \def\Ph      {\ensuremath{h}\xspace}
 \def\Pi      {\ensuremath{i}\xspace}
 \def\Pj      {\ensuremath{j}\xspace}
 \def\Pk      {\ensuremath{k}\xspace}
 \def\Pl      {\ensuremath{l}\xspace}
 \def\Pm      {\ensuremath{m}\xspace}
 \def\Pn      {\ensuremath{n}\xspace}
 \def\Po      {\ensuremath{o}\xspace}
 \def\Pp      {\ensuremath{p}\xspace}
 \def\Pq      {\ensuremath{q}\xspace}
 \def\Pr      {\ensuremath{r}\xspace}
 \def\Ps      {\ensuremath{s}\xspace}
 \def\Pt      {\ensuremath{t}\xspace}
 \def\Pu      {\ensuremath{u}\xspace}
 \def\Pv      {\ensuremath{v}\xspace}
 \def\Pw      {\ensuremath{w}\xspace}
 \def\Px      {\ensuremath{x}\xspace}
 \def\Py      {\ensuremath{y}\xspace}
 \def\Pz      {\ensuremath{z}\xspace}
 \def\thebaroffset{0.18em}
}
\newcommand{\offsetoverline}[2][\thebaroffset]{\kern #1\overline{\kern -#1 #2}}%

%%%%%%%%%%%%%%%%%%%%%%%%%%%%%%%%%%%%%%%%%%%%%%%
% Particles
\makeatletter
%% for my beamer then ptsize is 9pt so
\newcommand{\miniscule}{\@setfontsize\miniscule{3}{4}}% \tiny: 4/5
%\ifcase \@ptsize \relax% 10pt
  %\newcommand{\miniscule}{\@setfontsize\miniscule{4}{5}}% \tiny: 5/6
%\or% 11pt
  %\newcommand{\miniscule}{\@setfontsize\miniscule{5}{6}}% \tiny: 6/7
%\or% 12pt
  %\newcommand{\miniscule}{\@setfontsize\miniscule{5}{6}}% \tiny: 6/7
%\fi
\makeatother


\DeclareRobustCommand{\optbar}[1]{\shortstack{{\miniscule (\rule[.5ex]{1.25em}{.18mm})}
  \\ [-.7ex] $#1$}}


%% Leptons

\let\emi\en
\def\electron   {{\ensuremath{\Pe}}\xspace}
\def\en         {{\ensuremath{\Pe^-}}\xspace}   % electron negative (\em is taken)
\def\ep         {{\ensuremath{\Pe^+}}\xspace}
\def\epm        {{\ensuremath{\Pe^\pm}}\xspace}
\def\emp        {{\ensuremath{\Pe^\mp}}\xspace}
\def\epem       {{\ensuremath{\Pe^+\Pe^-}}\xspace}
%%%\def\ee         {\ensuremath{\Pe^-\Pe^-}\xspace}

\def\muon       {{\ensuremath{\Pmu}}\xspace}
\def\mup        {{\ensuremath{\Pmu^+}}\xspace}
\def\mun        {{\ensuremath{\Pmu^-}}\xspace} % muon negative (\mum is taken)
\def\mupm       {{\ensuremath{\Pmu^\pm}}\xspace}
\def\mump       {{\ensuremath{\Pmu^\mp}}\xspace}
\def\mumu       {{\ensuremath{\Pmu^+\Pmu^-}}\xspace}

\def\tauon      {{\ensuremath{\Ptau}}\xspace}
\def\taup       {{\ensuremath{\Ptau^+}}\xspace}
\def\taum       {{\ensuremath{\Ptau^-}}\xspace}
\def\taupm      {{\ensuremath{\Ptau^\pm}}\xspace}
\def\taump      {{\ensuremath{\Ptau^\mp}}\xspace}
\def\tautau     {{\ensuremath{\Ptau^+\Ptau^-}}\xspace}

\def\lepton     {{\ensuremath{\ell}}\xspace}
\def\ellm       {{\ensuremath{\ell^-}}\xspace}
\def\ellp       {{\ensuremath{\ell^+}}\xspace}
\def\ellpm      {{\ensuremath{\ell^\pm}}\xspace}
\def\ellmp      {{\ensuremath{\ell^\mp}}\xspace}
\def\ellell     {\ensuremath{\ell^+ \ell^-}\xspace}

\def\neu        {{\ensuremath{\Pnu}}\xspace}
\def\neub       {{\ensuremath{\overline{\Pnu}}}\xspace}
%%%\def\nuenueb    {\ensuremath{\neu\neub}\xspace}
\def\neue       {{\ensuremath{\neu_e}}\xspace}
\def\neueb      {{\ensuremath{\neub_e}}\xspace}
%%%\def\neueneueb  {\ensuremath{\neue\neueb}\xspace}
\def\neum       {{\ensuremath{\neu_\mu}}\xspace}
\def\neumb      {{\ensuremath{\neub_\mu}}\xspace}
%%%\def\neumneumb  {\ensuremath{\neum\neumb}\xspace}
\def\neut       {{\ensuremath{\neu_\tau}}\xspace}
\def\neutb      {{\ensuremath{\neub_\tau}}\xspace}
%%%\def\neutneutb  {\ensuremath{\neut\neutb}\xspace}
\def\neul       {{\ensuremath{\neu_\ell}}\xspace}
\def\neulb      {{\ensuremath{\neub_\ell}}\xspace}
%%%\def\neulneulb  {\ensuremath{\neul\neulb}\xspace}

%% Gauge bosons and scalars

\def\g      {{\ensuremath{\Pgamma}}\xspace}
\def\H      {{\ensuremath{\PH^0}}\xspace}
\def\Hp     {{\ensuremath{\PH^+}}\xspace}
\def\Hm     {{\ensuremath{\PH^-}}\xspace}
\def\Hpm    {{\ensuremath{\PH^\pm}}\xspace}
\def\W      {{\ensuremath{\PW}}\xspace}
\def\Wp     {{\ensuremath{\PW^+}}\xspace}
\def\Wm     {{\ensuremath{\PW^-}}\xspace}
\def\Wpm    {{\ensuremath{\PW^\pm}}\xspace}
\def\Z      {{\ensuremath{\PZ}}\xspace}

%% Quarks

\def\quark     {{\ensuremath{\Pq}}\xspace}
\def\quarkbar  {{\ensuremath{\overline \quark}}\xspace}
\def\qqbar     {{\ensuremath{\quark\quarkbar}}\xspace}
\def\uquark    {{\ensuremath{\Pu}}\xspace}
\def\uquarkbar {{\ensuremath{\overline \uquark}}\xspace}
\def\uubar     {{\ensuremath{\uquark\uquarkbar}}\xspace}
\def\dquark    {{\ensuremath{\Pd}}\xspace}
\def\dquarkbar {{\ensuremath{\overline \dquark}}\xspace}
\def\ddbar     {{\ensuremath{\dquark\dquarkbar}}\xspace}
\def\squark    {{\ensuremath{\Ps}}\xspace}
\def\squarkbar {{\ensuremath{\overline \squark}}\xspace}
\def\ssbar     {{\ensuremath{\squark\squarkbar}}\xspace}
\def\cquark    {{\ensuremath{\Pc}}\xspace}
\def\cquarkbar {{\ensuremath{\overline \cquark}}\xspace}
\def\ccbar     {{\ensuremath{\cquark\cquarkbar}}\xspace}
\def\bquark    {{\ensuremath{\Pb}}\xspace}
\def\bquarkbar {{\ensuremath{\overline \bquark}}\xspace}
\def\bbbar     {{\ensuremath{\bquark\bquarkbar}}\xspace}
\def\tquark    {{\ensuremath{\Pt}}\xspace}
\def\tquarkbar {{\ensuremath{\overline \tquark}}\xspace}
\def\ttbar     {{\ensuremath{\tquark\tquarkbar}}\xspace}

%% Light mesons

\def\hadron {{\ensuremath{\Ph}}\xspace}
\def\pion   {{\ensuremath{\Ppi}}\xspace}
\def\piz    {{\ensuremath{\pion^0}}\xspace}
\def\pip    {{\ensuremath{\pion^+}}\xspace}
\def\pim    {{\ensuremath{\pion^-}}\xspace}
\def\pipm   {{\ensuremath{\pion^\pm}}\xspace}
\def\pimp   {{\ensuremath{\pion^\mp}}\xspace}

\def\rhomeson {{\ensuremath{\Prho}}\xspace}
\def\rhoz     {{\ensuremath{\rhomeson^0}}\xspace}
\def\rhop     {{\ensuremath{\rhomeson^+}}\xspace}
\def\rhom     {{\ensuremath{\rhomeson^-}}\xspace}
\def\rhopm    {{\ensuremath{\rhomeson^\pm}}\xspace}
\def\rhomp    {{\ensuremath{\rhomeson^\mp}}\xspace}

\def\kaon    {{\ensuremath{\PK}}\xspace}
%%% do NOT use ensuremath here, and keep indent
\def\Kbar    {{\ensuremath{\offsetoverline{\PK}}}\xspace}
\def\Kb      {{\ensuremath{\Kbar}}\xspace}
\def\KorKbar {\kern \thebaroffset\optbar{\kern -\thebaroffset \PK}{}\xspace}
\def\Kz      {{\ensuremath{\kaon^0}}\xspace}
\def\Kzb     {{\ensuremath{\Kbar{}^0}}\xspace}
\def\Kp      {{\ensuremath{\kaon^+}}\xspace}
\def\Km      {{\ensuremath{\kaon^-}}\xspace}
\def\Kpm     {{\ensuremath{\kaon^\pm}}\xspace}
\def\Kmp     {{\ensuremath{\kaon^\mp}}\xspace}
\def\KS      {{\ensuremath{\kaon^0_{\mathrm{S}}}}\xspace}
\def\Vzero   {{\ensuremath{V^0}}\xspace}
\def\KL      {{\ensuremath{\kaon^0_{\mathrm{L}}}}\xspace}
\def\Kstarz  {{\ensuremath{\kaon^{*0}}}\xspace}
\def\Kstarzb {{\ensuremath{\Kbar{}^{*0}}}\xspace}
\def\Kstar   {{\ensuremath{\kaon^*}}\xspace}
\def\Kstarb  {{\ensuremath{\Kbar{}^*}}\xspace}
\def\Kstarp  {{\ensuremath{\kaon^{*+}}}\xspace}
\def\Kstarm  {{\ensuremath{\kaon^{*-}}}\xspace}
\def\Kstarpm {{\ensuremath{\kaon^{*\pm}}}\xspace}
\def\Kstarmp {{\ensuremath{\kaon^{*\mp}}}\xspace}
\def\KorKbarz {\ensuremath{\KorKbar^0}\xspace}

\newcommand{\etaz}{\ensuremath{\Peta}\xspace}
\newcommand{\etapr}{\ensuremath{\Peta^{\prime}}\xspace}
\newcommand{\phiz}{\ensuremath{\Pphi}\xspace}
\newcommand{\omegaz}{\ensuremath{\Pomega}\xspace}

%% Charmed mesons

%%% do NOT use ensuremath here (and keep indent)
\def\Dbar    {{\ensuremath{\offsetoverline{\PD}}}\xspace}
\def\D       {{\ensuremath{\PD}}\xspace}
\def\Db      {{\ensuremath{\Dbar}}\xspace}
\def\DorDbar {\kern \thebaroffset\optbar{\kern -\thebaroffset \PD}\xspace}
\def\Dz      {{\ensuremath{\D^0}}\xspace}
\def\Dzb     {{\ensuremath{\Dbar{}^0}}\xspace}
\def\Dp      {{\ensuremath{\D^+}}\xspace}
\def\Dm      {{\ensuremath{\D^-}}\xspace}
\def\Dpm     {{\ensuremath{\D^\pm}}\xspace}
\def\Dmp     {{\ensuremath{\D^\mp}}\xspace}
\def\DpDm    {\ensuremath{\Dp {\kern -0.16em \Dm}}\xspace}
\def\Dstar   {{\ensuremath{\D^*}}\xspace}
\def\Dstarb  {{\ensuremath{\Dbar{}^*}}\xspace}
\def\Dstarz  {{\ensuremath{\D^{*0}}}\xspace}
\def\Dstarzb {{\ensuremath{\Dbar{}^{*0}}}\xspace}
\def\theDstarz{{\ensuremath{\D^{*}(2007)^{0}}}\xspace}
\def\theDstarzb{{\ensuremath{\Dbar^{*}(2007)^{0}}}\xspace}
\def\Dstarp  {{\ensuremath{\D^{*+}}}\xspace}
\def\Dstarm  {{\ensuremath{\D^{*-}}}\xspace}
\def\Dstarpm {{\ensuremath{\D^{*\pm}}}\xspace}
\def\Dstarmp {{\ensuremath{\D^{*\mp}}}\xspace}
\def\theDstarp{{\ensuremath{\D^{*}(2010)^{+}}}\xspace}
\def\theDstarm{{\ensuremath{\D^{*}(2010)^{-}}}\xspace}
\def\theDstarpm{{\ensuremath{\D^{*}(2010)^{\pm}}}\xspace}
\def\theDstarmp{{\ensuremath{\D^{*}(2010)^{\mp}}}\xspace}
\def\Ds      {{\ensuremath{\D^+_\squark}}\xspace}
\def\Dsp     {{\ensuremath{\D^+_\squark}}\xspace}
\def\Dsm     {{\ensuremath{\D^-_\squark}}\xspace}
\def\Dspm    {{\ensuremath{\D^{\pm}_\squark}}\xspace}
\def\Dsmp    {{\ensuremath{\D^{\mp}_\squark}}\xspace}
\def\Dss     {{\ensuremath{\D^{*+}_\squark}}\xspace}
\def\Dssp    {{\ensuremath{\D^{*+}_\squark}}\xspace}
\def\Dssm    {{\ensuremath{\D^{*-}_\squark}}\xspace}
\def\Dsspm   {{\ensuremath{\D^{*\pm}_\squark}}\xspace}
\def\Dssmp   {{\ensuremath{\D^{*\mp}_\squark}}\xspace}
\def\DporDsp {{\ensuremath{\D_{(\squark)}^+}}\xspace}
\def\DmorDsm {{\ensuremath{\D{}_{(\squark)}^-}}\xspace}
\def\DpmorDspm {{\ensuremath{\D{}_{(\squark)}^\pm}}\xspace}

%% Beauty mesons
\def\B       {{\ensuremath{\PB}}\xspace}
\def\Bbar    {{\ensuremath{\offsetoverline{\PB}}}\xspace}
\def\Bb      {{\ensuremath{\Bbar}}\xspace}
\def\BorBbar {\kern \thebaroffset\optbar{\kern -\thebaroffset \PB}\xspace}
\def\Bz      {{\ensuremath{\B^0}}\xspace}
\def\Bzb     {{\ensuremath{\Bbar{}^0}}\xspace}
\def\Bd      {{\ensuremath{\B^0}}\xspace}
\def\Bdb     {{\ensuremath{\Bbar{}^0}}\xspace}
\def\BdorBdbar {\kern \thebaroffset\optbar{\kern -\thebaroffset \Bd}\xspace}
\def\Bu      {{\ensuremath{\B^+}}\xspace}
\def\Bub     {{\ensuremath{\B^-}}\xspace}
\def\Bp      {{\ensuremath{\Bu}}\xspace}
\def\Bm      {{\ensuremath{\Bub}}\xspace}
\def\Bpm     {{\ensuremath{\B^\pm}}\xspace}
\def\Bmp     {{\ensuremath{\B^\mp}}\xspace}
\def\Bs      {{\ensuremath{\B^0_\squark}}\xspace}
\def\Bsb     {{\ensuremath{\Bbar{}^0_\squark}}\xspace}
\def\BsorBsbar {\kern \thebaroffset\optbar{\kern -\thebaroffset \Bs}\xspace}
\def\Bc      {{\ensuremath{\B_\cquark^+}}\xspace}
\def\Bcp     {{\ensuremath{\B_\cquark^+}}\xspace}
\def\Bcm     {{\ensuremath{\B_\cquark^-}}\xspace}
\def\Bcpm    {{\ensuremath{\B_\cquark^\pm}}\xspace}
\def\Bds     {{\ensuremath{\B_{(\squark)}^0}}\xspace}
\def\Bdsb    {{\ensuremath{\Bbar{}_{(\squark)}^0}}\xspace}
\def\BdorBs  {\Bds}
\def\BdorBsbar  {\Bdsb}

%% Onia

\def\jpsi     {{\ensuremath{{\PJ\mskip -3mu/\mskip -2mu\Ppsi}}}\xspace}
\def\psitwos  {{\ensuremath{\Ppsi{(2S)}}}\xspace}
\def\psiprpr  {{\ensuremath{\Ppsi(3770)}}\xspace}
\def\etac     {{\ensuremath{\Peta_\cquark}}\xspace}
\def\psires  {{\ensuremath{\Ppsi}}\xspace}
\def\chic     {{\ensuremath{\Pchi_\cquark}}\xspace}
\def\chiczero {{\ensuremath{\Pchi_{\cquark 0}}}\xspace}
\def\chicone  {{\ensuremath{\Pchi_{\cquark 1}}}\xspace}
\def\chictwo  {{\ensuremath{\Pchi_{\cquark 2}}}\xspace}
\def\chicJ    {{\ensuremath{\Pchi_{\cquark J}}}\xspace}
\def\Upsilonres  {{\ensuremath{\PUpsilon}}\xspace}
\def\Y#1S{\ensuremath{\PUpsilon{(#1S)}}\xspace}
\def\OneS  {{\Y1S}}
\def\TwoS  {{\Y2S}}
\def\ThreeS{{\Y3S}}
\def\FourS {{\Y4S}}
\def\FiveS {{\Y5S}}
\def\chib     {{\ensuremath{\Pchi_{b}}}\xspace}
\def\chibzero {{\ensuremath{\Pchi_{\bquark 0}}}\xspace}
\def\chibone  {{\ensuremath{\Pchi_{\bquark 1}}}\xspace}
\def\chibtwo  {{\ensuremath{\Pchi_{\bquark 2}}}\xspace}
\def\chibJ    {{\ensuremath{\Pchi_{\bquark J}}}\xspace}
\def\theX     {{\ensuremath{\Pchi_{c1}(3872)}}\xspace}

%% Light Baryons

\def\proton      {{\ensuremath{\Pp}}\xspace}
\def\antiproton  {{\ensuremath{\overline \proton}}\xspace}
\def\neutron     {{\ensuremath{\Pn}}\xspace}
\def\antineutron {{\ensuremath{\overline \neutron}}\xspace}
\def\Deltares    {{\ensuremath{\PDelta}}\xspace}
\def\Deltaresbar {{\ensuremath{\overline \Deltares}}\xspace}
%%% uds singlet
\def\Lz          {{\ensuremath{\PLambda}}\xspace}
\def\Lbar        {{\ensuremath{\offsetoverline{\PLambda}}}\xspace}
\def\LorLbar     {\kern \thebaroffset\optbar{\kern -\thebaroffset \PLambda}\xspace}
\def\Lambdares   {{\ensuremath{\PLambda}}\xspace}
\def\Lambdaresbar{{\ensuremath{\Lbar}}\xspace}
%%% uus, uds, dds
\def\Sigmares    {{\ensuremath{\PSigma}}\xspace}
\def\Sigmaz      {{\ensuremath{\Sigmares{}^0}}\xspace}
\def\Sigmap      {{\ensuremath{\Sigmares{}^+}}\xspace}
\def\Sigmam      {{\ensuremath{\Sigmares{}^-}}\xspace}
\def\Sigmaresbar {{\ensuremath{\offsetoverline{\Sigmares}}}\xspace}
\def\Sigmabarz   {{\ensuremath{\Sigmaresbar{}^0}}\xspace}
\def\Sigmabarp   {{\ensuremath{\Sigmaresbar{}^+}}\xspace}
\def\Sigmabarm   {{\ensuremath{\Sigmaresbar{}^-}}\xspace}
%%%  uss, dss
\def\Xires       {{\ensuremath{\PXi}}\xspace}
\def\Xiz         {{\ensuremath{\Xires^0}}\xspace}
\def\Xim         {{\ensuremath{\Xires^-}}\xspace}
\def\Xiresbar       {{\ensuremath{\offsetoverline{\Xires}}}\xspace}
\def\Xibarz      {{\ensuremath{\Xiresbar^0}}\xspace}
\def\Xibarp      {{\ensuremath{\Xiresbar^+}}\xspace}
%%%  sss
\def\Omegares    {{\ensuremath{\POmega}}\xspace}
\def\Omegaresbar {{\ensuremath{\offsetoverline{\POmega}}}\xspace}
\def\Omegam      {{\ensuremath{\Omegares^-}}\xspace}
\def\Omegabarp   {{\ensuremath{\Omegaresbar^+}}\xspace}

%% Charmed Baryons
\def\Lc          {{\ensuremath{\Lz^+_\cquark}}\xspace}
\def\Lcbar       {{\ensuremath{\Lbar{}^-_\cquark}}\xspace}
\def\Xic         {{\ensuremath{\Xires_\cquark}}\xspace}
\def\Xicz        {{\ensuremath{\Xires^0_\cquark}}\xspace}
\def\Xicp        {{\ensuremath{\Xires^+_\cquark}}\xspace}
\def\Xicbar      {{\ensuremath{\Xiresbar{}_\cquark}}\xspace}
\def\Xicbarz     {{\ensuremath{\Xiresbar{}_\cquark^0}}\xspace}
\def\Xicbarm     {{\ensuremath{\Xiresbar{}_\cquark^-}}\xspace}
\def\Omegac      {{\ensuremath{\Omegares^0_\cquark}}\xspace}
\def\Omegacbar   {{\ensuremath{\Omegaresbar{}_\cquark^0}}\xspace}
\def\Xicc        {{\ensuremath{\Xires_{\cquark\cquark}}}\xspace}
\def\Xiccbar     {{\ensuremath{\Xiresbar{}_{\cquark\cquark}}}\xspace}
\def\Xiccp       {{\ensuremath{\Xires^+_{\cquark\cquark}}}\xspace}
\def\Xiccpp      {{\ensuremath{\Xires^{++}_{\cquark\cquark}}}\xspace}
\def\Xiccbarm    {{\ensuremath{\Xiresbar{}_{\cquark\cquark}^-}}\xspace}
\def\Xiccbarmm   {{\ensuremath{\Xiresbar{}_{\cquark\cquark}^{--}}}\xspace}
\def\Omegacc     {{\ensuremath{\Omegares^+_{\cquark\cquark}}}\xspace}
\def\Omegaccbar  {{\ensuremath{\Omegaresbar{}_{\cquark\cquark}^-}}\xspace}
\def\Omegaccc    {{\ensuremath{\Omegares^{++}_{\cquark\cquark\cquark}}}\xspace}
\def\Omegacccbar {{\ensuremath{\Omegaresbar{}_{\cquark\cquark\cquark}^{--}}}\xspace}
%% Beauty Baryons

\def\Lb           {{\ensuremath{\Lz^0_\bquark}}\xspace}
\def\Lbbar        {{\ensuremath{\Lbar{}^0_\bquark}}\xspace}
\def\Sigmab       {{\ensuremath{\Sigmares_\bquark}}\xspace}
\def\Sigmabp      {{\ensuremath{\Sigmares_\bquark^+}}\xspace}
\def\Sigmabz      {{\ensuremath{\Sigmares_\bquark^0}}\xspace}
\def\Sigmabm      {{\ensuremath{\Sigmares_\bquark^-}}\xspace}
\def\Sigmabpm     {{\ensuremath{\Sigmares_\bquark^\pm}}\xspace}
\def\Sigmabbar    {{\ensuremath{\Sigmaresbar_\bquark}}\xspace}
\def\Sigmabbarp   {{\ensuremath{\Sigmaresbar_\bquark^+}}\xspace}
\def\Sigmabbarz   {{\ensuremath{\Sigmaresbar_\bquark^0}}\xspace}
\def\Sigmabbarm   {{\ensuremath{\Sigmaresbar_\bquark^-}}\xspace}
\def\Sigmabbarpm  {{\ensuremath{\Sigmaresbar_\bquark^-}}\xspace}
\def\Xib          {{\ensuremath{\Xires_\bquark}}\xspace}
\def\Xibz         {{\ensuremath{\Xires^0_\bquark}}\xspace}
\def\Xibm         {{\ensuremath{\Xires^-_\bquark}}\xspace}
\def\Xibbar       {{\ensuremath{\Xiresbar{}_\bquark}}\xspace}
\def\Xibbarz      {{\ensuremath{\Xiresbar{}_\bquark^0}}\xspace}
\def\Xibbarp      {{\ensuremath{\Xiresbar{}_\bquark^+}}\xspace}
\def\Omegab       {{\ensuremath{\Omegares^-_\bquark}}\xspace}
\def\Omegabbar    {{\ensuremath{\Omegaresbar{}_\bquark^+}}\xspace}

%%%%%%%%%%%%%%%%%%
% Physics symbols
%%%%%%%%%%%%%%%%%

%% Decays
\def\BF         {{\ensuremath{\mathcal{B}}}\xspace}
\def\BR         {\BF}
\def\BRvis      {{\ensuremath{\BR_{\mathrm{{vis}}}}}}
\newcommand{\decay}[2]{\ensuremath{#1\!\to #2}\xspace}
\def\ra                 {\ensuremath{\rightarrow}\xspace}
\def\to                 {\ensuremath{\rightarrow}\xspace}

%% Lifetimes
\newcommand{\tauBs}{{\ensuremath{\tau_{\Bs}}}\xspace}
\newcommand{\tauBd}{{\ensuremath{\tau_{\Bd}}}\xspace}
\newcommand{\tauBz}{{\ensuremath{\tau_{\Bz}}}\xspace}
\newcommand{\tauBu}{{\ensuremath{\tau_{\Bp}}}\xspace}
\newcommand{\tauDp}{{\ensuremath{\tau_{\Dp}}}\xspace}
\newcommand{\tauDz}{{\ensuremath{\tau_{\Dz}}}\xspace}
\newcommand{\tauL}{{\ensuremath{\tau_{\mathrm{ L}}}}\xspace}
\newcommand{\tauH}{{\ensuremath{\tau_{\mathrm{ H}}}}\xspace}

%% Masses
\newcommand{\mBd}{{\ensuremath{m_{\Bd}}}\xspace}
\newcommand{\mBp}{{\ensuremath{m_{\Bp}}}\xspace}
\newcommand{\mBs}{{\ensuremath{m_{\Bs}}}\xspace}
\newcommand{\mBc}{{\ensuremath{m_{\Bc}}}\xspace}
\newcommand{\mLb}{{\ensuremath{m_{\Lb}}}\xspace}

%% EW theory, groups
\def\grpsuthree {{\ensuremath{\mathrm{SU}(3)}}\xspace}
\def\grpsutw    {{\ensuremath{\mathrm{SU}(2)}}\xspace}
\def\grpuone    {{\ensuremath{\mathrm{U}(1)}}\xspace}

\def\ssqtw   {{\ensuremath{\sin^{2}\!\theta_{\mathrm{W}}}}\xspace}
\def\csqtw   {{\ensuremath{\cos^{2}\!\theta_{\mathrm{W}}}}\xspace}
\def\stw     {{\ensuremath{\sin\theta_{\mathrm{W}}}}\xspace}
\def\ctw     {{\ensuremath{\cos\theta_{\mathrm{W}}}}\xspace}
\def\ssqtwef {{\ensuremath{{\sin}^{2}\theta_{\mathrm{W}}^{\mathrm{eff}}}}\xspace}
\def\csqtwef {{\ensuremath{{\cos}^{2}\theta_{\mathrm{W}}^{\mathrm{eff}}}}\xspace}
\def\stwef   {{\ensuremath{\sin\theta_{\mathrm{W}}^{\mathrm{eff}}}}\xspace}
\def\ctwef   {{\ensuremath{\cos\theta_{\mathrm{W}}^{\mathrm{eff}}}}\xspace}
\def\gv      {{\ensuremath{g_{\mbox{\tiny V}}}}\xspace}
\def\ga      {{\ensuremath{g_{\mbox{\tiny A}}}}\xspace}

\def\order   {{\ensuremath{\mathcal{O}}}\xspace}
\def\ordalph {{\ensuremath{\mathcal{O}(\alpha)}}\xspace}
\def\ordalsq {{\ensuremath{\mathcal{O}(\alpha^{2})}}\xspace}
\def\ordalcb {{\ensuremath{\mathcal{O}(\alpha^{3})}}\xspace}

%% QCD parameters
\newcommand{\as}{{\ensuremath{\alpha_s}}\xspace}
\newcommand{\MSb}{{\ensuremath{\overline{\mathrm{MS}}}}\xspace}
\newcommand{\lqcd}{{\ensuremath{\Lambda_{\mathrm{QCD}}}}\xspace}
\def\qsq       {{\ensuremath{q^2}}\xspace}

%% CKM, \boldmath \CP violation

\def\eps   {{\ensuremath{\varepsilon}}\xspace}
\def\epsK  {{\ensuremath{\varepsilon_K}}\xspace}
\def\epsB  {{\ensuremath{\varepsilon_B}}\xspace}
\def\epsp  {{\ensuremath{\varepsilon^\prime_K}}\xspace}

\def\CP                {{\ensuremath{C\!P}}\xspace}
\def\CPT               {{\ensuremath{C\!PT}}\xspace}
\def\T                 {{\ensuremath{T}}\xspace}

\def\rhobar {{\ensuremath{\overline \rho}}\xspace}
\def\etabar {{\ensuremath{\overline \eta}}\xspace}

\def\Vud  {{\ensuremath{V_{\uquark\dquark}^{\phantom{\ast}}}}\xspace}
\def\Vcd  {{\ensuremath{V_{\cquark\dquark}^{\phantom{\ast}}}}\xspace}
\def\Vtd  {{\ensuremath{V_{\tquark\dquark}^{\phantom{\ast}}}}\xspace}
\def\Vus  {{\ensuremath{V_{\uquark\squark}^{\phantom{\ast}}}}\xspace}
\def\Vcs  {{\ensuremath{V_{\cquark\squark}^{\phantom{\ast}}}}\xspace}
\def\Vts  {{\ensuremath{V_{\tquark\squark}^{\phantom{\ast}}}}\xspace}
\def\Vub  {{\ensuremath{V_{\uquark\bquark}^{\phantom{\ast}}}}\xspace}
\def\Vcb  {{\ensuremath{V_{\cquark\bquark}^{\phantom{\ast}}}}\xspace}
\def\Vtb  {{\ensuremath{V_{\tquark\bquark}^{\phantom{\ast}}}}\xspace}
\def\Vuds  {{\ensuremath{V_{\uquark\dquark}^\ast}}\xspace}
\def\Vcds  {{\ensuremath{V_{\cquark\dquark}^\ast}}\xspace}
\def\Vtds  {{\ensuremath{V_{\tquark\dquark}^\ast}}\xspace}
\def\Vuss  {{\ensuremath{V_{\uquark\squark}^\ast}}\xspace}
\def\Vcss  {{\ensuremath{V_{\cquark\squark}^\ast}}\xspace}
\def\Vtss  {{\ensuremath{V_{\tquark\squark}^\ast}}\xspace}
\def\Vubs  {{\ensuremath{V_{\uquark\bquark}^\ast}}\xspace}
\def\Vcbs  {{\ensuremath{V_{\cquark\bquark}^\ast}}\xspace}
\def\Vtbs  {{\ensuremath{V_{\tquark\bquark}^\ast}}\xspace}

%% Oscillations

\newcommand{\dm}{{\ensuremath{\Delta m}}\xspace}
\newcommand{\dms}{{\ensuremath{\Delta m_{\squark}}}\xspace}
\newcommand{\dmd}{{\ensuremath{\Delta m_{\dquark}}}\xspace}
\newcommand{\DG}{{\ensuremath{\Delta\Gamma}}\xspace}
\newcommand{\DGs}{{\ensuremath{\Delta\Gamma_{\squark}}}\xspace}
\newcommand{\DGd}{{\ensuremath{\Delta\Gamma_{\dquark}}}\xspace}
\newcommand{\Gs}{{\ensuremath{\Gamma_{\squark}}}\xspace}
\newcommand{\Gd}{{\ensuremath{\Gamma_{\dquark}}}\xspace}
\newcommand{\MBq}{{\ensuremath{M_{\B_\quark}}}\xspace}
\newcommand{\DGq}{{\ensuremath{\Delta\Gamma_{\quark}}}\xspace}
\newcommand{\Gq}{{\ensuremath{\Gamma_{\quark}}}\xspace}
\newcommand{\dmq}{{\ensuremath{\Delta m_{\quark}}}\xspace}
\newcommand{\GL}{{\ensuremath{\Gamma_{\mathrm{ L}}}}\xspace}
\newcommand{\GH}{{\ensuremath{\Gamma_{\mathrm{ H}}}}\xspace}
\newcommand{\DGsGs}{{\ensuremath{\Delta\Gamma_{\squark}/\Gamma_{\squark}}}\xspace}
\newcommand{\Delm}{{\mbox{$\Delta m $}}\xspace}
\newcommand{\ACP}{{\ensuremath{{\mathcal{A}}^{\CP}}}\xspace}
\newcommand{\Adir}{{\ensuremath{{\mathcal{A}}^{\mathrm{ dir}}}}\xspace}
\newcommand{\Amix}{{\ensuremath{{\mathcal{A}}^{\mathrm{ mix}}}}\xspace}
\newcommand{\ADelta}{{\ensuremath{{\mathcal{A}}^\Delta}}\xspace}
\newcommand{\phid}{{\ensuremath{\phi_{\dquark}}}\xspace}
\newcommand{\sinphid}{{\ensuremath{\sin\!\phid}}\xspace}
\newcommand{\phis}{{\ensuremath{\phi_{\squark}}}\xspace}
\newcommand{\betas}{{\ensuremath{\beta_{\squark}}}\xspace}
\newcommand{\sbetas}{{\ensuremath{\sigma(\beta_{\squark})}}\xspace}
\newcommand{\stbetas}{{\ensuremath{\sigma(2\beta_{\squark})}}\xspace}
\newcommand{\stphis}{{\ensuremath{\sigma(\phi_{\squark})}}\xspace}
\newcommand{\sinphis}{{\ensuremath{\sin\!\phis}}\xspace}

%% Tagging
\newcommand{\edet}{{\ensuremath{\varepsilon_{\mathrm{ det}}}}\xspace}
\newcommand{\erec}{{\ensuremath{\varepsilon_{\mathrm{ rec/det}}}}\xspace}
\newcommand{\esel}{{\ensuremath{\varepsilon_{\mathrm{ sel/rec}}}}\xspace}
\newcommand{\etrg}{{\ensuremath{\varepsilon_{\mathrm{ trg/sel}}}}\xspace}
\newcommand{\etot}{{\ensuremath{\varepsilon_{\mathrm{ tot}}}}\xspace}

\newcommand{\mistag}{\ensuremath{\omega}\xspace}
\newcommand{\wcomb}{\ensuremath{\omega^{\mathrm{comb}}}\xspace}
\newcommand{\etag}{{\ensuremath{\varepsilon_{\mathrm{tag}}}}\xspace}
\newcommand{\etagcomb}{{\ensuremath{\varepsilon_{\mathrm{tag}}^{\mathrm{comb}}}}\xspace}
\newcommand{\effeff}{\ensuremath{\varepsilon_{\mathrm{eff}}}\xspace}
\newcommand{\effeffcomb}{\ensuremath{\varepsilon_{\mathrm{eff}}^{\mathrm{comb}}}\xspace}
\newcommand{\efftag}{{\ensuremath{\etag(1-2\omega)^2}}\xspace}
\newcommand{\effD}{{\ensuremath{\etag D^2}}\xspace}

\newcommand{\etagprompt}{{\ensuremath{\varepsilon_{\mathrm{ tag}}^{\mathrm{Pr}}}}\xspace}
\newcommand{\etagLL}{{\ensuremath{\varepsilon_{\mathrm{ tag}}^{\mathrm{LL}}}}\xspace}

%% Key decay channels

\def\BdToKstmm    {\decay{\Bd}{\Kstarz\mup\mun}}
\def\BdbToKstmm   {\decay{\Bdb}{\Kstarzb\mup\mun}}

\def\BsToJPsiPhi  {\decay{\Bs}{\jpsi\phi}}
\def\BdToJPsiKst  {\decay{\Bd}{\jpsi\Kstarz}}
\def\BdbToJPsiKst {\decay{\Bdb}{\jpsi\Kstarzb}}

\def\BsPhiGam     {\decay{\Bs}{\phi \g}}
\def\BdKstGam     {\decay{\Bd}{\Kstarz \g}}

\def\BTohh        {\decay{\B}{\Ph^+ \Ph'^-}}
\def\BdTopipi     {\decay{\Bd}{\pip\pim}}
\def\BdToKpi      {\decay{\Bd}{\Kp\pim}}
\def\BsToKK       {\decay{\Bs}{\Kp\Km}}
\def\BsTopiK      {\decay{\Bs}{\pip\Km}}
\def\Cpipi        {\ensuremath{C_{\pip\pim}}\xspace}
\def\Spipi        {\ensuremath{S_{\pip\pim}}\xspace}
\def\CKK          {\ensuremath{C_{\Kp\Km}}\xspace}
\def\SKK          {\ensuremath{S_{\Kp\Km}}\xspace}
\def\ADGKK        {\ensuremath{A^{\DG}_{\Kp\Km}}\xspace}

%% Rare decays
\def\BdKstee  {\decay{\Bd}{\Kstarz\epem}}
\def\BdbKstee {\decay{\Bdb}{\Kstarzb\epem}}
\def\bsll     {\decay{\bquark}{\squark \ell^+ \ell^-}}
\def\AFB      {\ensuremath{A_{\mathrm{FB}}}\xspace}
\def\FL       {\ensuremath{F_{\mathrm{L}}}\xspace}
\def\AT#1     {\ensuremath{A_{\mathrm{T}}^{#1}}\xspace}           % 2
\def\btosgam  {\decay{\bquark}{\squark \g}}
\def\btodgam  {\decay{\bquark}{\dquark \g}}
\def\Bsmm     {\decay{\Bs}{\mup\mun}}
\def\Bdmm     {\decay{\Bd}{\mup\mun}}
\def\Bsee     {\decay{\Bs}{\epem}}
\def\Bdee     {\decay{\Bd}{\epem}}
\def\ctl       {\ensuremath{\cos{\theta_\ell}}\xspace}
\def\ctk       {\ensuremath{\cos{\theta_K}}\xspace}

%% Wilson coefficients and operators
\def\C#1      {\ensuremath{\mathcal{C}_{#1}}\xspace}                       % 9
\def\Cp#1     {\ensuremath{\mathcal{C}_{#1}^{'}}\xspace}                    % 7
\def\Ceff#1   {\ensuremath{\mathcal{C}_{#1}^{\mathrm{(eff)}}}\xspace}        % 9
\def\Cpeff#1  {\ensuremath{\mathcal{C}_{#1}^{'\mathrm{(eff)}}}\xspace}       % 7
\def\Ope#1    {\ensuremath{\mathcal{O}_{#1}}\xspace}                       % 2
\def\Opep#1   {\ensuremath{\mathcal{O}_{#1}^{'}}\xspace}                    % 7

%% Charm

\def\xprime     {\ensuremath{x^{\prime}}\xspace}
\def\yprime     {\ensuremath{y^{\prime}}\xspace}
\def\ycp        {\ensuremath{y_{\CP}}\xspace}
\def\agamma     {\ensuremath{A_{\Gamma}}\xspace}
%%%\def\kpi        {\ensuremath{\PK\Ppi}\xspace}
%%%\def\kk         {\ensuremath{\PK\PK}\xspace}
%%%\def\dkpi       {\decay{\PD}{\PK\Ppi}}
%%%\def\dkk        {\decay{\PD}{\PK\PK}}
\def\dkpicf     {\decay{\Dz}{\Km\pip}}

%% QM
\newcommand{\bra}[1]{\ensuremath{\langle #1|}}             % {a}
\newcommand{\ket}[1]{\ensuremath{|#1\rangle}}              % {b}
\newcommand{\braket}[2]{\ensuremath{\langle #1|#2\rangle}} % {a}{b}

%%%%%%%%%%%%%%%%%%%%%%%%%%%%%%%%%%%%%%%%%%%%%%%%%%
% Units (these macros add a small space in front)
%%%%%%%%%%%%%%%%%%%%%%%%%%%%%%%%%%%%%%%%%%%%%%%%%%
\newcommand{\nospaceunit}[1]{\ensuremath{\text{#1}}}
\newcommand{\aunit}[1]{\ensuremath{\text{\,#1}}}
\newcommand{\unit}[1]{\aunit{#1}\xspace}                   % {kg}

%% Energy and momentum
\newcommand{\tev}{\aunit{Te\kern -0.1em V}\xspace}
\newcommand{\gev}{\aunit{Ge\kern -0.1em V}\xspace}
\newcommand{\mev}{\aunit{Me\kern -0.1em V}\xspace}
\newcommand{\kev}{\aunit{ke\kern -0.1em V}\xspace}
\newcommand{\ev}{\aunit{e\kern -0.1em V}\xspace}
\newcommand{\gevgev}{\ensuremath{\gev^2}\xspace}
\newcommand{\mevc}{\ensuremath{\aunit{Me\kern -0.1em V\!/}c}\xspace}
\newcommand{\gevc}{\ensuremath{\aunit{Ge\kern -0.1em V\!/}c}\xspace}
\newcommand{\mevcc}{\ensuremath{\aunit{Me\kern -0.1em V\!/}c^2}\xspace}
\newcommand{\gevcc}{\ensuremath{\aunit{Ge\kern -0.1em V\!/}c^2}\xspace}
\newcommand{\gevgevcc}{\ensuremath{\gev^2\!/c^2}\xspace} % for \pt^2 in CEP
\newcommand{\gevgevcccc}{\ensuremath{\gev^2\!/c^4}\xspace} % for q^2

%% Distance and area (these macros add a small space)
\def\km   {\aunit{km}\xspace}
\def\m    {\aunit{m}\xspace}
\def\ma   {\ensuremath{\aunit{m}^2}\xspace}
\def\cm   {\aunit{cm}\xspace}
\def\cma  {\ensuremath{\aunit{cm}^2}\xspace}
\def\mm   {\aunit{mm}\xspace}
\def\mma  {\ensuremath{\aunit{mm}^2}\xspace}
\def\mum  {\ensuremath{\,\upmu\nospaceunit{m}}\xspace}
\def\muma {\ensuremath{\,\upmu\nospaceunit{m}^2}\xspace}
\def\nm   {\aunit{nm}\xspace}
\def\fm   {\aunit{fm}\xspace}
\def\barn{\aunit{b}\xspace}
%%%\def\barnhyph{\ensuremath{\mathrm{ -b}}
\def\mbarn{\aunit{mb}\xspace}
\def\mub{\ensuremath{\,\upmu\nospaceunit{b}}\xspace}
%%%\def\mbarnhyph{\ensuremath{\mathrm{ -mb}}
\def\nb {\aunit{nb}\xspace}
\def\invnb {\ensuremath{\nb^{-1}}\xspace}
\def\pb {\aunit{pb}\xspace}
\def\invpb {\ensuremath{\pb^{-1}}\xspace}
\def\fb   {\ensuremath{\aunit{fb}}\xspace}
\def\invfb   {\ensuremath{\fb^{-1}}\xspace}
\def\ab   {\ensuremath{\aunit{ab}}\xspace}
\def\invab   {\ensuremath{\ab^{-1}}\xspace}

%% Time
\def\sec  {\ensuremath{\aunit{s}}\xspace}
\def\ms   {\ensuremath{\aunit{ms}}\xspace}
\def\mus  {\ensuremath{\,\upmu\nospaceunit{s}}\xspace}
\def\ns   {\ensuremath{\aunit{ns}}\xspace}
\def\ps   {\ensuremath{\aunit{ps}}\xspace}
\def\fs   {\aunit{fs}}

\def\mhz  {\ensuremath{\aunit{MHz}}\xspace}
\def\khz  {\ensuremath{\aunit{kHz}}\xspace}
\def\hz   {\ensuremath{\aunit{Hz}}\xspace}

\def\invps{\ensuremath{\ps^{-1}}\xspace}
\def\invns{\ensuremath{\ns^{-1}}\xspace}

\def\yr   {\aunit{yr}\xspace}
\def\hr   {\aunit{hr}\xspace}

%% Temperature
\def\degc {\ensuremath{^\circ}{\text{C}}\xspace}
\def\degk {\aunit{K}\xspace}

%% Material lengths, radiation
\def\Xrad {\ensuremath{X_0}\xspace}
\def\NIL{\ensuremath{\lambda_{\rm int}}\xspace}
\def\mip {MIP\xspace}
\def\neutroneq {\ensuremath{n_\nospaceunit{eq}}\xspace}
\def\neqcmcm {\ensuremath{\neutroneq/\nospaceunit{cm}^2}\xspace}
\def\kRad {\aunit{kRad}\xspace}
\def\MRad {\aunit{MRad}\xspace}
\def\ci {\aunit{Ci}\xspace}
\def\mci {\aunit{mCi}\xspace}

%% Uncertainties
\def\sx    {\ensuremath{\sigma_x}\xspace}
\def\sy    {\ensuremath{\sigma_y}\xspace}
\def\sz    {\ensuremath{\sigma_z}\xspace}

\newcommand{\stat}{\aunit{(stat)}\xspace}
\newcommand{\syst}{\aunit{(syst)}\xspace}
\newcommand{\lumi}{\aunit{(lumi)}\xspace}

%% Maths

\def\order{{\ensuremath{\mathcal{O}}}\xspace}
\newcommand{\chisq}{\ensuremath{\chi^2}\xspace}
\newcommand{\chisqndf}{\ensuremath{\chi^2/\mathrm{ndf}}\xspace}
\newcommand{\chisqip}{\ensuremath{\chi^2_{\text{IP}}}\xspace}
\newcommand{\chisqfd}{\ensuremath{\chi^2_{\text{FD}}}\xspace}
\newcommand{\chisqvs}{\ensuremath{\chi^2_{\text{VS}}}\xspace}
\newcommand{\chisqvtx}{\ensuremath{\chi^2_{\text{vtx}}}\xspace}
\newcommand{\chisqvtxndf}{\ensuremath{\chi^2_{\text{vtx}}/\mathrm{ndf}}\xspace}

\def\deriv {\ensuremath{\mathrm{d}}}

\def\gsim{{~\raise.15em\hbox{$>$}\kern-.85em
          \lower.35em\hbox{$\sim$}~}\xspace}
\def\lsim{{~\raise.15em\hbox{$<$}\kern-.85em
          \lower.35em\hbox{$\sim$}~}\xspace}

\newcommand{\mean}[1]{\ensuremath{\left\langle #1 \right\rangle}} % {x}
\newcommand{\abs}[1]{\ensuremath{\left\|#1\right\|}} % {x}
\newcommand{\Real}{\ensuremath{\mathcal{R}e}\xspace}
\newcommand{\Imag}{\ensuremath{\mathcal{I}m}\xspace}

\def\PDF {PDF\xspace}

\def\sPlot{\mbox{\em sPlot}\xspace}
\def\sFit{\mbox{\em sFit}\xspace}
%%%\def\sWeight{\mbox{\em sWeight}\xspace}

%%%%%%%%%%%%%%%%%%%%%%%%%%%%%%%%%%%%%%%%%%%%%%%%%%
% Kinematics
%%%%%%%%%%%%%%%%%%%%%%%%%%%%%%%%%%%%%%%%%%%%%%%%%%

%% Energy, Momenta
\def\Ebeam {\ensuremath{E_{\mbox{\tiny BEAM}}}\xspace}
\def\sqs   {\ensuremath{\protect\sqrt{s}}\xspace}
\def\sqsnn {\ensuremath{\protect\sqrt{s_{\scriptscriptstyle\text{NN}}}}\xspace}
\def\pt         {\ensuremath{p_{\mathrm{T}}}\xspace}
\def\ptsq       {\ensuremath{p_{\mathrm{T}}^2}\xspace}
\def\ptot       {\ensuremath{p}\xspace}
\def\et         {\ensuremath{E_{\mathrm{T}}}\xspace}
\def\mt         {\ensuremath{M_{\mathrm{T}}}\xspace}
\def\dpp        {\ensuremath{\Delta p/p}\xspace}
\def\msq        {\ensuremath{m^2}\xspace}
\newcommand{\dedx}{\ensuremath{\mathrm{d}\hspace{-0.1em}E/\mathrm{d}x}\xspace}
%% PID
\def\dllkpi     {\ensuremath{\mathrm{DLL}_{\kaon\pion}}\xspace}
\def\dllppi     {\ensuremath{\mathrm{DLL}_{\proton\pion}}\xspace}
\def\dllepi     {\ensuremath{\mathrm{DLL}_{\electron\pion}}\xspace}
\def\dllmupi    {\ensuremath{\mathrm{DLL}_{\muon\pi}}\xspace}
%% Geometry
%%%\def\mphi       {\mbox{$\phi$}\xspace}
%%%\def\mtheta     {\mbox{$\theta$}\xspace}
%%%\def\ctheta     {\mbox{$\cos\theta$}\xspace}
%%%\def\stheta     {\mbox{$\sin\theta$}\xspace}
%%%\def\ttheta     {\mbox{$\tan\theta$}\xspace}

\def\degrees{\ensuremath{^{\circ}}\xspace}
\def\murad{\ensuremath{\,\upmu\nospaceunit{rad}}\xspace}
\def\mrad{\aunit{mrad}\xspace}
\def\rad{\aunit{rad}\xspace}

%% Accelerator
\def\betastar {\ensuremath{\beta^*}}
\newcommand{\lum} {\ensuremath{\mathcal{L}}\xspace}
\newcommand{\intlum}[1]{\ensuremath{\int\lum=#1}\xspace}  % {2 \,\invfb}

%%%%%%%%%%%%%%%%%%%%%%%%%%%%%%%%%%%%%%%%%%%%%%%%%%%%%%%%%%%%%%%%%%%%
% Software
%%%%%%%%%%%%%%%%%%%%%%%%%%%%%%%%%%%%%%%%%%%%%%%%%%%%%%%%%%%%%%%%%%%%

%% Programs
%%%\def\ansys      {\mbox{\textsc{Ansys}}\xspace}
\def\bcvegpy    {\mbox{\textsc{Bcvegpy}}\xspace}
\def\boole      {\mbox{\textsc{Boole}}\xspace}
\def\brunel     {\mbox{\textsc{Brunel}}\xspace}
\def\davinci    {\mbox{\textsc{DaVinci}}\xspace}
\def\dirac      {\mbox{\textsc{Dirac}}\xspace}
%%%\def\erasmus    {\mbox{\textsc{Erasmus}}\xspace}
\def\evtgen     {\mbox{\textsc{EvtGen}}\xspace}
\def\fewz       {\mbox{\textsc{Fewz}}\xspace}
\def\fluka      {\mbox{\textsc{Fluka}}\xspace}
\def\ganga      {\mbox{\textsc{Ganga}}\xspace}
%%%\def\garfield   {\mbox{\textsc{Garfield}}\xspace}
\def\gaudi      {\mbox{\textsc{Gaudi}}\xspace}
\def\gauss      {\mbox{\textsc{Gauss}}\xspace}
\def\geant      {\mbox{\textsc{Geant4}}\xspace}
\def\hepmc      {\mbox{\textsc{HepMC}}\xspace}
\def\herwig     {\mbox{\textsc{Herwig}}\xspace}
\def\moore      {\mbox{\textsc{Moore}}\xspace}
\def\neurobayes {\mbox{\textsc{NeuroBayes}}\xspace}
\def\photos     {\mbox{\textsc{Photos}}\xspace}
\def\powheg     {\mbox{\textsc{Powheg}}\xspace}
%%%\def\pyroot     {\mbox{\textsc{PyRoot}}\xspace}
\def\pythia     {\mbox{\textsc{Pythia}}\xspace}
\def\resbos     {\mbox{\textsc{ResBos}}\xspace}
\def\roofit     {\mbox{\textsc{RooFit}}\xspace}
\def\root       {\mbox{\textsc{Root}}\xspace}
\def\spice      {\mbox{\textsc{Spice}}\xspace}
%%%\def\tosca      {\mbox{\textsc{Tosca}}\xspace}
\def\urania     {\mbox{\textsc{Urania}}\xspace}

%% Languages
\def\cpp        {\mbox{\textsc{C\raisebox{0.1em}{{\footnotesize{++}}}}}\xspace}
%%%\def\python     {\mbox{\textsc{Python}}\xspace}
\def\ruby       {\mbox{\textsc{Ruby}}\xspace}
\def\fortran    {\mbox{\textsc{Fortran}}\xspace}
\def\svn        {\mbox{\textsc{svn}}\xspace}
\def\git        {\mbox{\textsc{git}}\xspace}
\def\latex      {\mbox{\LaTeX}\xspace}

%% Data processing
\def\kbit          {\aunit{kbit}\xspace}
\def\kbps        {\aunit{kbit/s}\xspace}
\def\kbytes     {\aunit{kB}\xspace}
\def\kbyps      {\aunit{kB/s}\xspace}
\def\mbit          {\aunit{Mbit}\xspace}
\def\mbps        {\aunit{Mbit/s}\xspace}
\def\mbytes     {\aunit{MB}\xspace}
\def\mbyps      {\aunit{MB/s}\xspace}
\def\gbit          {\aunit{Gbit}\xspace}
\def\gbps        {\aunit{Gbit/s}\xspace}
\def\gbytes     {\aunit{GB}\xspace}
\def\gbyps      {\aunit{GB/s}\xspace}
\def\tbit          {\aunit{Tbit}\xspace}
\def\tbps        {\aunit{Tbit/s}\xspace}
\def\tbytes     {\aunit{TB}\xspace}
\def\tbyps      {\aunit{TB/s}\xspace}
\def\dst        {DST\xspace}

%%%%%%%%%%%%%%%%%%%%%%%%%%%
% Detector related
%%%%%%%%%%%%%%%%%%%%%%%%%%%

%% Detector technologies
\def\nonn {\ensuremath{\mathrm{{ \mathit{n^+}} \mbox{-} on\mbox{-}{ \mathit{n}}}}\xspace}
\def\ponn {\ensuremath{\mathrm{{ \mathit{p^+}} \mbox{-} on\mbox{-}{ \mathit{n}}}}\xspace}
\def\nonp {\ensuremath{\mathrm{{ \mathit{n^+}} \mbox{-} on\mbox{-}{ \mathit{p}}}}\xspace}
\def\cvd  {CVD\xspace}
\def\mwpc {MWPC\xspace}
\def\gem  {GEM\xspace}

%% Detector components, electronics
\def\tell1  {TELL1\xspace}
\def\ukl1   {UKL1\xspace}
\def\beetle {Beetle\xspace}
\def\otis   {OTIS\xspace}
\def\croc   {CROC\xspace}
\def\carioca {CARIOCA\xspace}
\def\dialog {DIALOG\xspace}
\def\sync   {SYNC\xspace}
\def\cardiac {CARDIAC\xspace}
\def\gol    {GOL\xspace}
\def\vcsel  {VCSEL\xspace}
\def\ttc    {TTC\xspace}
\def\ttcrx  {TTCrx\xspace}
\def\hpd    {HPD\xspace}
\def\pmt    {PMT\xspace}
\def\specs  {SPECS\xspace}
\def\elmb   {ELMB\xspace}
\def\fpga   {FPGA\xspace}
\def\plc    {PLC\xspace}
\def\rasnik {RASNIK\xspace}
\def\elmb   {ELMB\xspace}
\def\can    {CAN\xspace}
\def\lvds   {LVDS\xspace}
\def\ntc    {NTC\xspace}
\def\adc    {ADC\xspace}
\def\led    {LED\xspace}
\def\ccd    {CCD\xspace}
\def\hv     {HV\xspace}
\def\lv     {LV\xspace}
\def\pvss   {PVSS\xspace}
\def\cmos   {CMOS\xspace}
\def\fifo   {FIFO\xspace}
\def\ccpc   {CCPC\xspace}

%% Chemical symbols
\def\cfourften     {\ensuremath{\mathrm{ C_4 F_{10}}}\xspace}
\def\cffour        {\ensuremath{\mathrm{ CF_4}}\xspace}
\def\cotwo         {\ensuremath{\mathrm{ CO_2}}\xspace}
\def\csixffouteen  {\ensuremath{\mathrm{ C_6 F_{14}}}\xspace}
\def\mgftwo     {\ensuremath{\mathrm{ Mg F_2}}\xspace}
\def\siotwo     {\ensuremath{\mathrm{ SiO_2}}\xspace}

%%%%%%%%%%%%%%%
% Special Text
%%%%%%%%%%%%%%%
\newcommand{\eg}{\mbox{\itshape e.g.}\xspace}
\newcommand{\ie}{\mbox{\itshape i.e.}\xspace}
\newcommand{\etal}{\mbox{\itshape et al.}\xspace}
\newcommand{\etc}{\mbox{\itshape etc.}\xspace}
\newcommand{\cf}{\mbox{\itshape cf.}\xspace}
\newcommand{\ffp}{\mbox{\itshape ff.}\xspace}
\newcommand{\vs}{\mbox{\itshape vs.}\xspace}
%%%%%%%%%%%%%%%
%% Helpful to align numbers in tables
%%%%%%%%%%%%%%%
\newcommand{\phz}{\phantom{0}}
%%%%%%%%%%%%%%%%%%%%%%%%%%%%%%%%%%%%%%%%%%%%%%%%%%%%%%%%%%%%%%%%%%%%%%%%
%%%                                                                    %
%%% !!!!!!!!!!!!!!!!!!! DO NOT EDIT THIS FILE !!!!!!!!!!!!!!!!!!!!!!!! %
%%%                                                                    %
%%% THE EB MAY OVERWRITE IT TO REFLECT LATEST CHANGES IN THE TEMPLATE  %
%%%                                                                    %
%%% You may define your own macros and packages in main.tex or add     %
%%% additional local files                                             %
%%%%%%%%%%%%%%%%%%%%%%%%%%%%%%%%%%%%%%%%%%%%%%%%%%%%%%%%%%%%%%%%%%%%%%%%

\begin{document}
\begin{axopicture}(190,140)

  % A meson                         
  \Text(30,92)[lb]{$\bquark$}       % anti-quark Label
  \Line[arrow](85,90)(25,90)        % anti-quark Line
  \Text(30,48)[lt]{$\squarkbar$}    % quark Label
  \Line[arrow](25,50)(85,50)        % quark Line
  \Text(18,70)[r]{$\Bsb$}           % Label
  \GOval(25,70)(5,20)(90){0.7}      % Bound State
                                    
  % Internal W line                 
  \Text(113,86){\small{$\Wm$}}      % Label
  \Photon(85,90)(125,70){-2}{5}     % Line
  \Vertex(85,90){2}                 % Start Vertex
  \Vertex(125,70){2}                % End Vertex
                                    
  % B meson                         
  \Text(160,129)[rb]{$\cquark$}     % anti-quark Label
  \Line[arrow](165,130)(85,90)      % anti-quark Line
  \Text(160,84)[rt]{$\cquarkbar$}   % quark Label
  \Line[arrow](125,70)(165,90)      % quark Line
  \Text(172,110)[l]{$\jpsi$}        % Label
  \GOval(165,110)(5,20)(90){0.7}    % Bound State
                                    
  % C meson                         
  \Text(160,56)[rb]{$\squark$}      % anti-quark Label
  \Line[arrow](165,50)(125,70)      % anti-quark Line
  \Text(160,11)[rt]{$\squarkbar$}   % quark Label
  \Line[arrow](85,50)(165,10)       % quark Line
  \Text(172,30)[l]{$\phi$}          % Label
  \GOval(165,30)(5,20)(90){0.7}     % Bound State
                                    
\end{axopicture}
\end{document}
