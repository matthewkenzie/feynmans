\documentclass{standalone}
\usepackage{axodraw2}
\usepackage{ifthen}
\newboolean{uprightparticles}
\setboolean{uprightparticles}{false}
\newboolean{pdflatex}
\setboolean{pdflatex}{true}
\input{lhcb-symbols-def}
\begin{document}
\begin{axopicture}(190,140)

  % A meson                                                
  \Text(30,92)[lb]{$\bquark$}                              % quark Label
  \Line[arrow](25,90)(85,90)                               % quark Line
  \Text(30,48)[lt]{$\squarkbar$}                           % anti-quark Label
  \Line[arrow](85,50)(25,50)                               % anti-quark Line
  \Text(18,70)[r]{$\Bsb$}                                  % Label
  \GOval(25,70)(5,20)(90){0.7}                             % Bound State
                                                           
  % Loop                                                   
  \Text(105,75){\small{$\Wm$}}                             % W Label
  \PhotonArc(108,86)(23,171,312){-3}{8}                    % W Line
                                                           % fermion Label
  \Arc[arrow,clockwise,arrowpos=0.35](102,74)(23,45,350)   % fermion Line1
  \Arc[arrow,clockwise,arrowpos=0.45](102,74)(23,135,45)   % fermion Line2
  \Vertex(85,90){2}                                        % Start Vertex
  \Vertex(125,70){2}                                       % End Vertex
  % Gluon                                                  
  \Gluon(113,95)(135,110){-3.5}{3}                         % gluon Line
  \Vertex(113,95){2}                                       % Start Vertex
  \Vertex(135,110){2}                                      % End Vertex
  % B meson                                                
  \Text(160,129)[rb]{$\cquark$}                            % quark Label
  \Line[arrow](135,110)(165,130)                           % quark Line
  \Text(160,91)[rt]{$\cquarkbar$}                          % anti-quark Label
  \Line[arrow](165,90)(135,110)                            % anti-quark Line
  \Text(172,110)[l]{$\jpsi$}                               % Label
  \GOval(165,110)(5,20)(90){0.7}                           % Bound State
                                                           
  % C meson                                                
  \Text(160,56)[rb]{$\squark$}                             % quark Label
  \Line[arrow](125,70)(165,50)                             % quark Line
  \Text(160,11)[rt]{$\squarkbar$}                          % anti-quark Label
  \Line[arrow](165,10)(85,50)                              % anti-quark Line
  \Text(172,30)[l]{$\phi$}                                 % Label
  \GOval(165,30)(5,20)(90){0.7}                            % Bound State
                                                           
\end{axopicture}
\end{document}
