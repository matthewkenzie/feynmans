\documentclass{standalone}
\usepackage{axodraw2}
\usepackage{ifthen}
\newboolean{uprightparticles}
\setboolean{uprightparticles}{false}
\newboolean{pdflatex}
\setboolean{pdflatex}{true}
\input{lhcb-symbols-def}
\begin{document}
\begin{axopicture}(190,140)

  % A meson                         
  \Text(30,92)[lb]{$\bquark$}       % quark Label
  \Line[arrow](25,90)(85,90)        % quark Line
  \Text(30,48)[lt]{$\uquarkbar$}    % anti-quark Label
  \Line[arrow](85,50)(25,50)        % anti-quark Line
  \Text(18,70)[r]{$\Bm$}            % Label
  \GOval(25,70)(5,20)(90){0.7}      % Bound State
                                    
  % External W line                 
  \Text(105,108){\small{$\Wm$}}     % Label
  \Photon(85,90)(125,110){-2}{5}    % Line
  \Vertex(85,90){2}                 % Start Vertex
  \Vertex(125,110){2}               % End Vertex
                                    
  % B meson                         
  \Text(160,129)[rb]{$\squark$}     % quark Label
  \Line[arrow](125,110)(165,130)    % quark Line
  \Text(160,91)[rt]{$\uquarkbar$}   % anti-quark Label
  \Line[arrow](165,90)(125,110)     % anti-quark Line
  \Text(172,110)[l]{$\Km$}          % Label
  \GOval(165,110)(5,20)(90){0.7}    % Bound State
                                    
  % C meson                         
  \Text(160,56)[rb]{$\cquark$}      % quark Label
  \Line[arrow](85,90)(165,50)       % quark Line
  \Text(160,11)[rt]{$\uquarkbar$}   % anti-quark Label
  \Line[arrow](165,10)(85,50)       % anti-quark Line
  \Text(172,30)[l]{$\Dz$}           % Label
  \GOval(165,30)(5,20)(90){0.7}     % Bound State
                                    
\end{axopicture}
\end{document}
